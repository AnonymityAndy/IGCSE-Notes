\documentclass[11pt, a4paper, openany]{book}
\usepackage{graphicx}
\usepackage{colortbl}
\usepackage{amsmath}
\usepackage{tikz}
\usepackage{blindtext}
\usepackage{microtype}
\usepackage{wrapfig}
\usepackage{enumitem}
\usepackage{fancyhdr}
\usepackage{index}
\usepackage{tcolorbox}
\usepackage[margin=1in]{geometry}
\usepackage[english]{babel}
%\usepackage{times}

\makeindex
\title{\textbf{IGCSE Economics Subject Content 1}}
\author{Ang Li, Frank}
\date{\today}

\begin{document}

\pagenumbering{roman}
\fancyhf{}
\renewcommand{\headrulewidth}{2pt}
\renewcommand{\footrulewidth}{1pt}
\fancyhead[LE]{\leftmark}
\fancyhead[RO]{\nouppercase{\rightmark}}
\fancyfoot[LE, RO]{\thepage}

\maketitle
\tableofcontents

\chapter{The Basic Economic Problem}

\section{The Nature of the Economic Problem}

\begin{tcolorbox}
\textbf{Key points from this section:}
\begin{enumerate}\itemsep0em
	\item Define the nature of the economic problem.
	\item Define the factors of production.
	\item Define opportunity cost and illustration of the concept.
	\item Demonstrate how production possibility curves can be used to illustrate choice and resource allocation.
	\item Evaluate the implications of particular courses of action in terms of opportunity cost.
\end{enumerate}
\end{tcolorbox}

\subsection{Finite resources and unlimited wants}

\begin{itemize}\itemsep0em
	\item \textbf{The basic economic problem:} How to allocate scarce resources to satisfy unlimited needs and wants.
	\item \textbf{Scarcity:} Unlimited wants and not enough limited resources to fulfill the wants.
	\item \textbf{Unlimited wants:} The unlimited desires that people have.
	\item \textbf{Limited resources:} The limited factors of production (resources) that we have on Earth.
\end{itemize}

\subsection{Economic and free goods}

<++>

\section{The Factors of Production}

<++>

\subsection{Definitions of the factors of productions and their rewards}

<++>

\subsection{Mobility of the factors of production}

<++>

\subsection{Quantity and quality of the factors of production}

<++>

\section{Opportunity Cost}

<++>

\subsection{Definition of opportunity cost}

<++>

\subsection{The influence of opportunity cost on decision making}

<++>

\section{Production Possibility Curve Diagrams (PPC)}

<++>

\subsection{Definition of PPC}

<++>

\subsection{Points under, on, and beyond a PPC}

<++>

\subsection{Movements along a PPC}

<++>

\subsection{Shifts in a PPC}

<++>

\end{document}
